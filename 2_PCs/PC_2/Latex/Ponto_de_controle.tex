% Template for FPL 2012 papers; to be used with:
%          spconf.sty   - ICASSP/ICIP LaTeX style file
%          IEEEtran.bst - IEEE bibliography style file

% Created:  Apr-May 2005 - Riku Uusikartano -- riku.uusikartano@tut.fi
% Modified: March-2012 - Daniel Mu�oz Arboleda -- damuz@unb.br
% --------------------------------------------------------------------------

\documentclass[10pt,a4paper]{article}

\usepackage{spconf,amsmath,epsfig}
\usepackage[brazilian]{babel} % Suporte para o Portugu�s
\usepackage[latin1]{inputenc} % Suporte para acentua��o sem necessidade dos comandos especiais.
\usepackage[]{subfigure}
\usepackage[portuguese,algoruled,longend]{algorithm2e}
\usepackage{multirow}
\usepackage{listings}


% Titulo do documento
% -------------------
\title{Ponto de controle 2 \\ Controle de acesso via reconhecimento de face humana}


% Nome dos autores
% ----------------
\name{
Ant�nio Ald�sio - 14/0130811 ---- Vitor Carvalho de Almeida - 14/0165380}
\address{Programa de Gradua��o em Engenharia Eletr�nica, Faculdade Gama\\
Universidade de Bras��lia\\
Gama, DF, Brasil\\\\
email: aldisiofilho@gmail.com ---- vitorcarvalhoamd@gmail.com }


\hyphenation{Tam-pe-re ela-bo-ra-cao}

\begin{document}

\maketitle

\begin{resumo}

O projeto consiste em construir um sistema de controle de acesso ativado por reconhecimento facial. Ser� poss��vel enviar os dados de acesso via rede para um banco de dados. Como valida��o ser� confeccionada uma porta em miniatura.

Neste ponto de controle s�o apresentadas as ferramentas, componentes, bibliotecas e m�dulos que ser�o usados no projeto. Tamb�m s�o mostradas as comunica��es da raspberry Pi com cada elemento.

\textbf{Palavras-chave:} Controle de acesso, Raspberry Pi, OpenCV, reconhecimento facial, seguran�a.

\end{resumo}
\section{Introdu��o}

\subsection{Justificativa}

O mundo encontra-se em uma grande evolu��o, nos dias atuais estamos de utilizar chaves para usar biometria, que � a mais difundida  a digital do dedo, por�m o usu�rio tem quer ter uma intera��o direta e tatio com o sistema para a sua  libera��o. Al�m da facilidade do uso e a impossibilidade de esquecer a chave de acesso, ou o cart�o, pode-se incluir um registro para utilizar como controle de ponto, ou adapitar esse sistema para fazer controle de produtividade em uma empresa.
Com base nessa tend�ncia e buscando uma facilidade para o usu�rio, diminui�o ao m�ximo do contato direito deste com o sistema. Portanto, esse artigo tenta construir esse sistema de reconhecimento facial de uma forma barata e confi�vel.

\subsection{Objetivos}

O objetivo desse artigo � a constru��o de um sistema de abertura de porta atrav�s do rosto do usu�rio cadastrado. Com a utiliza��o de uma raspberry pi, uma webcam RGB e uma c�mera infravermelho para o sistema e a constru��o de uma miniatura de porta para simula��o e valida��o.

\subsection{Requisitos}

A execu��o desse artigo � necess�ria a utiliza��o da biblioteca OPENCV, que � respons�vel por conseguir distinguir um rosto do resto da imagem, iremos utilizar ela na linguagem de C++.

\subsection{Benef�cios}

Um sistema de reconhecimento facial traz alguns benef�cios como: praticidade, seguran�a. No caso desenvolvimento o enfoque � a seguran�a e a possibilidade de utilizar essa valida��o de entrada como um ponto eletr�nico para contagem de horas trabalhadas
\section{Desenvolvimento}


\subsection{Descri��o do Hardware}

Foi montado um sistema de ativa��o da trava eletr�nica. Utilizando os seguintes materiais:

\begin{itemize}
\item Trava solenoide 12V (figura \ref{trava});
\item Fonte DC 12V ;
\item Resistor de 1 KOhm;
\item Transistor NPN (TIP41);
\item Jumpers
\item Protoboard
\end{itemize}


\begin{figure}[h!]
\caption{Trava eletr�nica solenoide 12V}
\centering % para centralizarmos a figura
\includegraphics[width=5cm]{trava.jpg} % leia abaixo
\label{trava}
\end{figure}

Na protoboard foi montado o circuito da figura \ref{circuito}.

\begin{figure}[h!]
\caption{Ativa��o da trava eletr�nica solenoide 12V}
\centering % para centralizarmos a figura
\includegraphics[width=5cm]{circuito_trava.jpg} % leia abaixo
\label{circuito}
\end{figure}

O pino de entrada foi conectado � GPIO4 da Raspberry Pi 3 para que fossem enviados os comandos para abrir a porta. 

A trava solenoide mantem a porta fechada at� que seja inserida uma tens�o de 12V em seus terminais. Neste momento, o solenoide faz com que o ''dente'' da trava seja retra�do, liberando a abertura da porta. Ao retirar a tens�o dos terminais, uma mola retorna a trava para a posi��o original, travando a porta novamente. [1]

Foi utilizada uma fonte DC de 12V - 2A com conex�o Jack P4, ligada na protoboard com um conector Jack P4 f�mea.

Foi conectada uma caixa de som � sa�da P2 da Raspberry Pi para reproduzir sons de confirma��o ou nega��o de acesso.


Para receber a requisi��o de acesso, foi montado um circuito com bot�o em modo Pull-Up, como mostra o esquematico da figura \ref{botao}

\begin{figure}[h!]
\caption{Bot�o em modo Pull-Up}
\centering % para centralizarmos a figura
\includegraphics[width=5cm]{circuito_botao.jpg} % leia abaixo
\label{botao}
\end{figure}

Foi utilizada uma c�mera com conex�o USB para testes (figura \ref{camera}).

\begin{figure}[h!]
\caption{C�mera utilizada para testes}
\centering % para centralizarmos a figura
\includegraphics[width=8cm]{camera.jpeg} % leia abaixo
\label{camera}
\end{figure}

\subsection{Descri��o do Software}

\subsubsection{Cadastro}

O projeto � composto de duas rotinas:  \textit{cadastrar usuario} e \textit{abertura}. Para realiza��o dessas rotinas ser� utilizada a raspberry como servidor e o cliente ser� um BOT no Telegram [2][3]. Pode-se observar as etapas do sistema de cadastro do ponto de vista do administrador na figura abaixo: 

\begin{figure}[h!]
\caption{Rotina Cadastrar}
\centering % para centralizarmos a figura
\includegraphics[width=8cm]{cadastro.jpg} % leia abaixo
\label{Rotina_cadastar}
\end{figure}

Na rotina \textit{cadastrar} ser� enviada uma foto via BOT Telegram, que pode ser tirada do celular ou computador do usu�rio e essa foto servir� como base para o sistema. Este ficar� aguardando o novo usu�rio pressionar o bot�o para ativar a c�mera presente na fachada, para ent�o poder tirar uma foto dele [4] e assim fazer uma verifica��o com a foto enviada e validar o cadastro. Ap�s isso, ser� enviada uma mensagem ao cliente (administrador) validando o cadastro do novo usuario.

Para tirar uma foto com a c�mera instalada na Raspberry Pi � utilizado o seguinte comando no terminal [4]:

\textit{fswebcam nome$\_$imagem.jpg}

Onde pode ser escolhido qualquer nome para a imagem.


\subsubsection{Acesso e resposta ao usu�rio}

Para a abertura da porta, ser� executada a rotina descrita da figura \ref{Rotina_Acesso}:

\begin{figure}[h!]
\caption{Rotina Acesso}
\centering % para centralizarmos a figura
\includegraphics[width=8cm]{abertura.jpg} % leia abaixo
\label{Rotina_Acesso}
\end{figure}

A GPIO3 foi configurada de modo a ficar em modo de espera, utilizando a fun��o \textit{polling}. Abaixo � apresentado o pseudo-c�digo da campainha. O c�digo completo do teste utilizado pode ser encontrado no ap�ndice.

\lstset{language=bash,
	numbers=left,
	linewidth=8cm,
	breaklines}
\begin{lstlisting}
Modo de espera
Bot�o foi pressionado? N-Espera S-Segue
Envia requisi��o de acesso para o c�digo principal.
\end{lstlisting}



A rotina para liberar a porta foi feita em um arquivo de instru��es bash. O c�digo \textit{abre.sh} � mostrado no apendice.

\begin{itemize}

\item[1]Primeiramente � reproduzido o som [5] de uma confirma��o de acesso pela caixa de som, para que o usu�rio saiba que pode entrar

\item[2] Depois a GPIO4 � definida como sa�da e colocada em n�vel l�gico alto.


\item[3] � definido um tempo de espera, no caso 3 segundos, para que a trava se mantenha retra�da e o usu�rio possa empurrar a porta.

\item[4] Ao final da contagem a GPIO4 volta para o n�vel l�gico baixo. Neste momento, ao encostar a porta, esta ser� trancada.

\item[5] Por fim, a GPIO4 � liberada para uso em outra rotina.

\end{itemize}


Caso o usu�rio n�o tenha acesso cadastrado, ao tocar a campainha deve ser executado o c�digo \textit{negado.sh}, mostrado abaixo.

\lstset{language=bash,
	numbers=left,
	linewidth=8cm,
	breaklines}
\begin{lstlisting}
#!/bin/bash

omxplayer -o local /home/pi/embarcados/projeto_final/sons/nao.mp3
\end{lstlisting}

Neste caso, a �nica fun��o realizada � a reprod��o de um som de nega��o na caixa de som, por�m na pr�tica, esse \textit{script} ser� chamado pelo servidor presente na Raspberry Pi. Ap�s a execu��o deste \textit{script}, o servidor enviar� para o cliente [6] a foto do sujeito que solicitou acesso.

Prevendo uma visita de algu�m de confian�a do administrador, por�m n�o cadastrada no sistema, ou eventuais falhas no reconhecimento facial, o administrador ter� a op��o de abrir a porta remotamente, via um comando master de acesso sem a necessidade de reconhecimento facial.


\subsubsection{Servidor}

O servidor � montando em cima da API disponilibilizada pelo pr�prio Telegram, e foi escrito na linguagem Python [2][3]. Para esse ponto de controle � realizado apenas a identifica��o dos texto "oi" e "ol�" com a resposta do bot "Ol� Mestre,o que deseja?" O codigo pode ser visto no ap�ndice 3.

\subsubsection{Reconhecimento facial}

Est� sendo utilizado as seguintes biliotecas:
\begin{itemize}
\item opencv 3.4.1 [7]
\item face$\_$recognition [8]
\end{itemize}

A primeira repons�vel pela vis�o computacional, ou seja, fazer o sistema identificar onde est� um rosto na imagem. J� o \textit{face$\_$recognition} � respons�vel pela compara��o do rosto com os cadastrados na base de dados.

As duas bibliotecas est�o na liguagem Python e para valida��o da factibilidade do projeto foi utilizado um exemplo da bibloteca \textit{face$\_$recognition}, cuja, faz a verifica��o quase inst�ntenia em um video ao vivo. O codigo pode ser visto no ap�ndice.

\section{Resultados}

O conjunto montado ficou como mostrado na figura \ref{montagem}:

\begin{figure}[h!]
\caption{Montagem do circuito}
\centering % para centralizarmos a figura
\includegraphics[width=8cm]{montagem.jpeg} % leia abaixo
\label{montagem}
\end{figure}


A ativa��o da trava eletr�nica foi realizada com sucesso, sem sobreaquecimento do transistor, nem falha na comunica��o.

Foi poss�vel comunicar o sistema com o cliente com sucesso, assim como executar os comandos recebidos, como mostram as figuras \ref{terminal} e \ref{terminal2}.

\begin{figure}[h!]
\caption{Cliente liberando porta, porta abrindo com e sem permiss�o}
\centering % para centralizarmos a figura
\includegraphics[width=8.5cm]{termin1.png} % leia abaixo
\label{terminal}
\end{figure}

Aqui, o primeiro comando do cliente � para liberar a  entrada. Pode-se observar que o programa s� passa a obedecer novamente quando a porta � fechada. NNa segunda instru��o, o cliente pede que o acesso seja bloqueado. E ent�o ao pressionar o bot�o, o sistema informa "Acesso negado". Ap�s isso foi simulada a abertura da porta por arrombamento, sem que o usu�rio fosse reconhecido. Imediatamente o alerta de invas�o foi ativado.

\begin{figure}[h!]
\caption{Cliente desativando alarme e liberando entrada}
\centering % para centralizarmos a figura
\includegraphics[width=8.5cm]{termin2.png} % leia abaixo
\label{terminal2}
\end{figure}

Aqui, a primeira instru��o do cliente � para desativar o alarme. Ocorrem atrasos at� que o alarme seja completamente desligado. Na segunda instru��o, o cliente bloqueia o acesso e com isso tamb�m ativa o alarme. Ao pressionar o bot�o, o acesso � negado. Ent�o o cliente envia o comando para liberar a porta, e ent�o ao pressionar o bot�o o acesso � liberado.
\section{Discuss�o e Conclus�es}


Pelos experimentos realizados, concluiu-se que o projeto pode ser implementado da forma como foi pensado inicialmente, pois tanto a comunica��o da Raspberry Pi com os perif�ricos (trava, bot�o, c�mera) quanto a biblioteca de reconhecimento facial do OpenCV funcionaram corretamente.

Decidiu-se usar o bot do Telegram como interface do administrador com o sistema, porque assim o acesso � muito mais pr�tico e podem ser recebidas notifica��es no celular sem necessidade de instala��o de outros aplicativos.

Ainda � necess�rio otimizar o uso do reconhecimento facial para que seja considerada a profundidade de campo, evitando que a porta seja aberta com a foto de um usu�rio cadastrado.

Para este documento, foram escritos programas cuja rotina principal � espec�fica para cada a��o. Por�m, para o pr�ximo ponto de controle, os c�digos ser�o unidos como subrotinas de um c�digo principal de controle do sistema.



\small
% IEEEtran is a LaTeX style file defining the reference formatting.
% -----------------------------------------------------------------
\section{Referencias}

 \begin{description}
 

\item[[1]] https://www.filipeflop.com/blog/acionando-trava-eletrica-com-rfid/

\item[[2]] https://pypi.org/project/pyTelegramBotAPI/0.2.9/

\item[[3]] https://core.telegram.org/bots/api

\item[[4]] https://www.raspberrypi.org/documentation/usage/webcams/README.md

\item[[5]] https://www.raspberrypi.org/documentation/usage/audio/README.md



\item[[6]] https://medium.com/@rosbots/ready-to-use-image-raspbian-stretch-ros-opencv-324d6f8dcd96

\item[[7]] https://github.com/opencv/opencv

\item[[8]] https://github.com/ageitgey/face$\_$recognition




 
  \end{description}

\section{Apendice}



C�digos utilizados

\noindent\rule{\columnwidth}{2pt}


Rotina do bot�o: \textit{campainha.c}

\lstset{language=C,
	numbers=left,
%	linewidth=10cm,
	breaklines}
\begin{lstlisting}
#include <stdio.h>
#include <stdlib.h>
#include <fcntl.h>
#include <sys/poll.h>
#include <unistd.h>

int main(void)
{
	struct pollfd pfd;
	char buffer;
	system("echo 3 > /sys/class/gpio/export");
	system("echo falling > /sys/class/gpio/gpio3/edge");
	system("echo in > /sys/class/gpio/gpio3/direction");
	pfd.fd = open("/sys/class/gpio/gpio3/value", O_RDONLY);
	if(pfd.fd < 0)
	{
		puts("Erro abrindo /sys/class/gpio/gpio3/value");
		puts("Execute este programa como root");
		return 1;
	}
	read(pfd.fd, &buffer, 1);
	pfd.events = POLLPRI | POLLERR;
	pfd.revents = 0;
	puts("Augardando campainha");
	poll(&pfd, 1, -1);
	if(pfd.revents) puts("Campainha pressionada: iniciar rotina de reconhecimento");
	close(pfd.fd);
	system("echo 3 > /sys/class/gpio/unexport");
	return 0;
}

\end{lstlisting}


\noindent\rule{\columnwidth}{2pt}


\textit{abre.sh}:



\lstset{language=bash,
	numbers=left,
	linewidth=8cm,
	breaklines}
\begin{lstlisting}
#!/bin/bash

GPIO_PATH=/sys/class/gpio

omxplayer -o local /home/pi/embarcados/projeto_final/sons/sim.mp3
echo 4 >> $GPIO_PATH/export
sudo echo out > $GPIO_PATH/gpio4/direction
sudo echo 1 > $GPIO_PATH/gpio4/value
sleep 3
echo 0 > $GPIO_PATH/gpio4/value
echo 4 >> $GPIO_PATH/unexport

\end{lstlisting}


\noindent\rule{\columnwidth}{2pt}


\textit{telegram.py}:
(C�digo do servidor para comunica��o com o bot do telegram)



\lstset{language=python,
	numbers=left,
	linewidth=8cm,
	breaklines}
\begin{lstlisting}
import sys
import time
import telepot
from telepot.loop import MessageLoop

def handle(msg):
    chat_id = msg['chat']['id']
    command =msg['text']
    print(command, chat_id)

#Inicio de conversa
    if command == 'Ola':
        bot.sendMessage(chat_id, 'Ol� Mestre, o que voc� deseja? ')
    if command == 'oi':
        bot.sendMessage(chat_id, 'Ol� Mestre, o que voc� deseja? ')
    if command == 'ola':
        bot.sendMessage(chat_id, 'Ol� Mestre, o que voc� deseja? ')
    if command == 'Oi':
        bot.sendMessage(chat_id, 'Ol� Mestre, o que voc� deseja? ')
# Cadastra novo usuario
    if command == 'Cadastrar':
        bot.sendMessage(chat_id, 'Informe o nome do usuario')

# Entrar
    if command == 'Abre':
        os.system('abre.sh')

    if command == 'Negado':
        os.system('negado.sh')

# Deletar usuario


bot = telepot.Bot('556806366:AAH9OIYZwwKapLkZrs7fYQU-NdUF2H9MuDc')
MessageLoop(bot, handle).run_as_thread()
print ('lendo ...')

# Keep the program running.
while 1:
    time.sleep(10)
    
\end{lstlisting}


\noindent\rule{\columnwidth}{2pt}


\textit{facerec$\_$from$\_$web.py}:
(C�digo do reconhecimento facial)



\lstset{language=python,
	numbers=left,
	linewidth=8cm,
	breaklines}
\begin{lstlisting}

import face_recognition
import cv2

# This is a demo of running face recognition on live video from your webcam. It's a little more complicated than the
# other example, but it includes some basic performance tweaks to make things run a lot faster:
#   1. Process each video frame at 1/4 resolution (though still display it at full resolution)
#   2. Only detect faces in every other frame of video.

# PLEASE NOTE: This example requires OpenCV (the `cv2` library) to be installed only to read from your webcam.
# OpenCV is *not* required to use the face_recognition library. It's only required if you want to run this
# specific demo. If you have trouble installing it, try any of the other demos that don't require it instead.

# Get a reference to webcam #0 (the default one)
video_capture = cv2.VideoCapture(0)

# Load a sample picture and learn how to recognize it.
obama_image = face_recognition.load_image_file("obama.jpg")
obama_face_encoding = face_recognition.face_encodings(obama_image)[0]

# Load a sample picture and learn how to recognize it.
luana_image = face_recognition.load_image_file("luana.jpg")
luana_face_encoding = face_recognition.face_encodings(luana_image)[0]

# Load a sample picture and learn how to recognize it.
vitinho_image = face_recognition.load_image_file("vitinho.jpg")
vitinho_face_encoding = face_recognition.face_encodings(vitinho_image)[0]

# Load a second sample picture and learn how to recognize it.

# Create arrays of known face encodings and their names
known_face_encodings = [
    obama_face_encoding,
    luana_face_encoding,
    vitinho_face_encoding
]
known_face_names = [
    "Barack Obama",
    "Luana",
    "vitinho"
]

# Initialize some variables
face_locations = []
face_encodings = []
face_names = []
process_this_frame = True

while True:
    # Grab a single frame of video
    ret, frame = video_capture.read()

    # Resize frame of video to 1/4 size for faster face recognition processing
    small_frame = cv2.resize(frame, (0, 0), fx=0.25, fy=0.25)

    # Convert the image from BGR color (which OpenCV uses) to RGB color (which face_recognition uses)
    rgb_small_frame = small_frame[:, :, ::-1]

    # Only process every other frame of video to save time
    if process_this_frame:
        # Find all the faces and face encodings in the current frame of video
        face_locations = face_recognition.face_locations(rgb_small_frame)
        face_encodings = face_recognition.face_encodings(rgb_small_frame, face_locations)

        face_names = []
        for face_encoding in face_encodings:
            # See if the face is a match for the known face(s)
            matches = face_recognition.compare_faces(known_face_encodings, face_encoding)
            name = "Unknown"

            # If a match was found in known_face_encodings, just use the first one.
            if True in matches:
                first_match_index = matches.index(True)
                name = known_face_names[first_match_index]

            face_names.append(name)

    process_this_frame = not process_this_frame


    # Display the results
    for (top, right, bottom, left), name in zip(face_locations, face_names):
        # Scale back up face locations since the frame we detected in was scaled to 1/4 size
        top *= 4
        right *= 4
        bottom *= 4
        left *= 4

        # Draw a box around the face
        cv2.rectangle(frame, (left, top), (right, bottom), (0, 0, 255), 2)

        # Draw a label with a name below the face
        cv2.rectangle(frame, (left, bottom - 35), (right, bottom), (0, 0, 255), cv2.FILLED)
        font = cv2.FONT_HERSHEY_DUPLEX
        cv2.putText(frame, name, (left + 6, bottom - 6), font, 1.0, (255, 255, 255), 1)

    # Display the resulting image
    cv2.imshow('Video', frame)

    # Hit 'q' on the keyboard to quit!
    if cv2.waitKey(1) & 0xFF == ord('q'):
        break

# Release handle to the webcam
video_capture.release()
cv2.destroyAllWindows()

\end{lstlisting}


\end{document}


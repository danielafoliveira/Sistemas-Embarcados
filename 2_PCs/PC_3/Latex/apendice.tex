\section{Apendice}



C�digos utilizados

\noindent\rule{\columnwidth}{2pt}

%%%%%%%%%%%%%%%%%%%%%%%%%%%%%%%%%%%%%%%%%%%%%%%%5

Fun��o principal: \textit{main.c}

\lstset{language=C,
	numbers=left,
%	linewidth=10cm,
	breaklines}
\begin{lstlisting}
#include <stdio.h>
#include <stdlib.h>
#include <fcntl.h>
#include <sys/poll.h>
#include <unistd.h>
#include <pthread.h>
#include <signal.h>
#include <wiringPi.h>
#include <sys/wait.h>
#include <string.h>

#include"funcoes.h"

int fim_curso = 0;
int campainha = 2;
int alarme = 3;

int porta_aberta = 0; //sensor fim de curso na porta
int reconhecido = 0; //sinal
int alerta = 0; //alarme de invas�o
int child_pid;

int encerrar=0;


pthread_t id_campainha;
pthread_t id_alarme;
pthread_t id_porta;

void encerra_prog(int sig);
void encerra_threads(int sig);
void* thread_campainha(void*arg);
void* thread_alarme(void*arg);
void* thread_porta(void*arg);

int main(int argc, char const *argv[]) {

  wiringPiSetup();
  pinMode(fim_curso, INPUT);
  pinMode(alarme, OUTPUT);

  if (fork() == 0){
      child_pid = getpid();
      signal(SIGINT,encerra_threads); //direcionando sinal de interrup��o (CTRL+C)

      pthread_create(&id_campainha,NULL,&thread_campainha,NULL); //criando thread para Campainha
      pthread_create(&id_alarme,NULL,&thread_alarme,NULL); //criando thread para Campainha
      pthread_create(&id_porta,NULL,&thread_porta,NULL); //criando thread para Campainha
	int a;

      char comando;
      while (!encerrar) {
        a = open("msgs_admin.txt", O_RDONLY);
	read(a,&comando,1);
	close(a);
        sleep(1);
        if (comando=='d'){
          alerta = 0;
        }
        if (comando=='l'){
          reconhecido = 1;
        }
        if (comando=='b'){
          reconhecido = 0;
        }
      }
  }

  if (fork()==0){//filho 2
      system("./servidor 8080"); // Executa o servidor
  }

  signal(SIGINT,encerra_prog); //direcionando sinal de interrup��o (CTRL+C)

  while (!encerrar);

  wait(NULL);
  wait(NULL);

  return 0;
}
void encerra_prog(int sig){
  encerrar = 1;

}
void encerra_threads(int sig){
  alerta = 0;

  puts("Encerrando...");
  encerrar = 1;
  if (pthread_cancel(id_campainha) ==-1){
    puts("tread da campainha nao foi cancelada");
  }
  if (pthread_cancel(id_alarme) ==-1){
    puts("tread do alarme nao foi cancelada");
  }

  if (pthread_cancel(id_porta) ==-1){
    puts("tread da port nao foi cancelada");
  }
printf("threads canceladas\n" );
  pthread_join(id_campainha,NULL);
  pthread_join(id_alarme,NULL);
  pthread_join(id_porta,NULL);

  system("gpio unexportall");
 puts("Programa encerrado pelo administrador!");

  exit(1);

}

void* thread_campainha(void*arg){
pthread_setcancelstate(PTHREAD_CANCEL_ENABLE, NULL);

  while (!encerrar) {

    poll_bot();
    if(porta_aberta == 0 && alerta== 0) { //s� inicia reconhecimento se a porta estiver fechada
      if (reconhecido == 1){ //se o usu�rio for cadastrado
        printf("Acesso permitido\n\n");
        system("sudo ./abre.sh");


        while(porta_aberta==1 && !encerrar){
            printf("porta aberta\n\n");
            sleep(1);
        } //espera porta fechar
        reconhecido = 0;
      }
      else{ //usu�rio nao cadastrado
        puts("Acesso negado\n\n");
        system("sudo ./negado.sh");

      }

    }
  }
  pthread_exit(0);

}

void* thread_alarme(void*arg){
  while(!encerrar){
      if(alerta == 1 && !encerrar){
          printf("Alerta de invasao\n\n");
          system("sudo ./alarme.sh");
          if (encerrar ==1){
            pthread_exit(0);
          }
      } //espera administrador desativar alarme;
    }
  pthread_exit(0);
}

void* thread_porta(void*arg){
  pthread_setcancelstate(PTHREAD_CANCEL_ENABLE, NULL);
  while(!encerrar){
    porta_aberta = digitalRead(fim_curso);
    if(porta_aberta == 1 && reconhecido == 0){
      alerta = 1;
      while (alerta == 1 && !encerrar);
    }
  }
  pthread_exit(0);

}
\end{lstlisting}
\noindent\rule{\columnwidth}{2pt}

\textit{servidor.c}

\lstset{language=C,
	numbers=left,
%	linewidth=10cm,
	breaklines}
\begin{lstlisting}
#include <stdio.h>
#include <stdlib.h>
#include <unistd.h>
#include <arpa/inet.h>
#include <string.h>
#include <signal.h>
#include <sys/socket.h>
#include <sys/un.h>

int socket_id;
void sigint_handler(int signum);
void print_client_message(int client_socket);
void end_server(void);

int main(int argc, char* const argv[]){

	unsigned short servidorPorta;
	struct sockaddr_in servidorAddr;

	servidorPorta = atoi(argv[1]);

//Definindo o tratamento de SIGINT
	signal(SIGINT, sigint_handler);


//	Abrindo o socket local
	socket_id = socket(PF_INET, SOCK_STREAM, IPPROTO_TCP);
	if(socket_id < 0)
	{
		fprintf(stderr, "Erro na criacao do socket!\n");
		exit(0);
	}


//Ligando o socket a porta
	memset(&servidorAddr, 0, sizeof(servidorAddr)); // Zerando a estrutura de dados
	servidorAddr.sin_family = AF_INET;
	servidorAddr.sin_addr.s_addr = htonl(INADDR_ANY);
	servidorAddr.sin_port = htons(servidorPorta);
	if(bind(socket_id, (struct sockaddr *) &servidorAddr, sizeof(servidorAddr)) < 0)
	{
		fprintf(stderr, "Erro na ligacao!\n");
		exit(0);
	}

//Tornando o socket passivo para virar um servidor
	if(listen(socket_id, 10) < 0)
	{
		fprintf(stderr, "Erro!\n");
		exit(0);
	}

	while(1)
	{
		int socketCliente;
		struct sockaddr_in clienteAddr;
		unsigned int clienteLength;

		fprintf(stderr, "Aguardando a conexao de um cliente...\n\n ");
		clienteLength = sizeof(clienteAddr);
		if((socketCliente = accept(socket_id, (struct sockaddr *) &clienteAddr, &clienteLength)) < 0)
			fprintf(stderr, "Falha no accept().\n");
		fprintf(stderr, "Feito!\n");

		fprintf(stderr, "Conex�o do Cliente %s\n", inet_ntoa(clienteAddr.sin_addr));

		fprintf(stderr, "Tratando comunicacao com o cliente... ");
		print_client_message(socketCliente);
		fprintf(stderr, "Feito!\n");

		fprintf(stderr, "Fechando a conexao com o cliente... ");
		close(socketCliente);
		fprintf(stderr, "Feito\n");
	}
	return 0;
}

void sigint_handler(int signum)
{
	fprintf(stderr, "\nRecebido o sinal CTRL+C... vamos desligar o servidor!\n");
	end_server();
}

void print_client_message(int client_socket)
{
  FILE *arq;
  arq = fopen("msgs_admin.txt", "wb");

	int length;
	char text;
	read(client_socket, &length, sizeof (length));
	read(client_socket, &text, 1);
  putc(text,arq); //Escreve no arquivo de transi��o;
  fclose(arq);
	if (text=='s')
	{	fprintf(stderr, "Cliente pediu para o servidor fechar.\n");
		end_server();
	}
}

void end_server(void)
{
	fprintf(stderr, "Fechando o socket local... ");
	close(socket_id);
	fprintf(stderr, "Feito!\n");
	exit(0);
}
\end{lstlisting}


\noindent\rule{\columnwidth}{2pt}

%%%%%%%%%%%%%%%%%%%%%%%%%%%%%%%%%%%%%%%%%%%%%%%%
\textit{abre.sh}:



\lstset{language=bash,
	numbers=left,
	linewidth=8cm,
	breaklines}
\begin{lstlisting}
#!/bin/bash

GPIO_PATH=/sys/class/gpio

omxplayer -o local /home/pi/embarcados/projeto_final/sons/sim.mp3
echo 4 >> $GPIO_PATH/export
sudo echo out > $GPIO_PATH/gpio4/direction
sudo echo 1 > $GPIO_PATH/gpio4/value
sleep 3
echo 0 > $GPIO_PATH/gpio4/value
echo 4 >> $GPIO_PATH/unexport

\end{lstlisting}


%%%%%%%%%%%%%%%%%%%%%%%%%%%%%%%%%%%%%%%%%%%%%%%%5

\noindent\rule{\columnwidth}{2pt}


\textit{negado.sh}:


\lstset{language=bash,
	numbers=left,
	linewidth=8cm,
	breaklines}
\begin{lstlisting}
#!/bin/bash
omxplayer -o local ./sons/nao.mp3
\end{lstlisting}


%%%%%%%%%%%%%%%%%%%%%%%%%%%%%%%%%%%%%%%%%%%%%%%%
\noindent\rule{\columnwidth}{2pt}
\textit{alarme.sh}:


\lstset{language=bash,
	numbers=left,
	linewidth=8cm,
	breaklines}
\begin{lstlisting}
#!/bin/bash
omxplayer -o local ./sons/alarme.mp3
\end{lstlisting}


%%%%%%%%%%%%%%%%%%%%%%%%%%%%%%%%%%%%%%%%%%%%%%%%5

\noindent\rule{\columnwidth}{2pt}
\textit{poll$\_$bot.c}

\lstset{language=C,
	numbers=left,
%	linewidth=10cm,
	breaklines}
\begin{lstlisting}
#include <stdio.h>
#include <stdlib.h>
#include <fcntl.h>
#include <sys/poll.h>
#include <unistd.h>

#include "funcoes.h"


int poll_bot()
{
	struct pollfd pfd;
	char buffer;
	system("echo 27 > /sys/class/gpio/export");
	system("echo falling > /sys/class/gpio/gpio27/edge");
	system("echo in > /sys/class/gpio/gpio27/direction");
	pfd.fd = open("/sys/class/gpio/gpio27/value", O_RDONLY);
	if(pfd.fd < 0)
	{
		puts("Erro abrindo /sys/class/gpio/gpio27/value");
		puts("Execute este programa como root");
		return -1;
	}
	read(pfd.fd, &buffer, 1);
	pfd.events = POLLPRI | POLLERR;
	pfd.revents = 0;
	puts("Augardando botao");
	poll(&pfd, 1, -1);
	if(pfd.revents) puts("mudanca do botao");
	usleep(500000);
	close(pfd.fd);
	system("echo 27 > /sys/class/gpio/unexport");
	return 0;
}

\end{lstlisting}

Obs: \textit{poll$\_$fim$\_$curso.c} � identico ao \textit{poll$\_$bot.c}, apenas trocando a GPIO 27 para a 17


\noindent\rule{\columnwidth}{2pt}


\textit{negado.sh}:


\lstset{language=bash,
	numbers=left,
	linewidth=8cm,
	breaklines}
\begin{lstlisting}
#!/bin/bash
omxplayer -o local ./sons/nao.mp3
\end{lstlisting}


%%%%%%%%%%%%%%%%%%%%%%%%%%%%%%%%%%%%%%%%%%%%%%%%
\noindent\rule{\columnwidth}{2pt}
\textit{alarme.sh}:


\lstset{language=bash,
	numbers=left,
	linewidth=8cm,
	breaklines}
\begin{lstlisting}
#!/bin/bash
omxplayer -o local ./sons/alarme.mp3
\end{lstlisting}


%%%%%%%%%%%%%%%%%%%%%%%%%%%%%%%%%%%%%%%%%%%%%%%%5

\noindent\rule{\columnwidth}{2pt}
\textit{face$\_$reco.py}

\lstset{language=python,
	numbers=left,
%	linewidth=10cm,
	breaklines}
\begin{lstlisting}
import face_recognition

known_image = face_recognition.load_image_file("imagens/vitinho.jpg")
unknown_image = face_recognition.load_image_file("imagens/unknown.jpg")

biden_encoding = face_recognition.face_encodings(known_image)[0]
unknown_encoding = face_recognition.face_encodings(unknown_image)[0]

results = face_recognition.compare_faces([biden_encoding], unknown_encoding)
arq = open('lista_verificacao.txt','w')
arq.write(str(results))
arq.close()
\end{lstlisting}

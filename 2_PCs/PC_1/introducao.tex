\section{Introdu��o}

\subsection{Justificativa}

O mundo encontra-se em uma grande evolu��o, nos dias atuais estamos de utilizar chaves para usar biometria, que � a mais difundida  a digital do dedo, por�m o usu�rio tem quer ter uma intera��o direta e tatio com o sistema para a sua  libera��o. Al�m da facilidade do uso e a impossibilidade de esquecer a chave de acesso, ou o cart�o, pode-se incluir um registro para utilizar como controle de ponto, ou adapitar esse sistema para fazer controle de produtividade em uma empresa.
Com base nessa tend�ncia e buscando uma facilidade para o usu�rio, diminui�o ao m�ximo do contato direito deste com o sistema. Portanto, esse artigo tenta construir esse sistema de reconhecimento facial de uma forma barata e confi�vel.

\subsection{Objetivos}

O objetivo desse artigo � a constru��o de um sistema de abertura de porta atrav�s do rosto do usu�rio cadastrado. Com a utiliza��o de uma raspberry pi, uma webcam RGB e uma c�mera infravermelho para o sistema e a constru��o de uma miniatura de porta para simula��o e valida��o.

\subsection{Requisitos}

A execu��o desse artigo � necess�ria a utiliza��o da biblioteca OPENCV, que � respons�vel por conseguir distinguir um rosto do resto da imagem, iremos utilizar ela na linguagem de C++.

\subsection{Benef�cios}

Um sistema de reconhecimento facial traz alguns benef�cios como: praticidade, seguran�a. No caso desenvolvimento o enfoque � a seguran�a e a possibilidade de utilizar essa valida��o de entrada como um ponto eletr�nico para contagem de horas trabalhadas
\section{Apendice}



C�digos utilizados

\noindent\rule{\columnwidth}{2pt}


Rotina do bot�o: \textit{campainha.c}

\lstset{language=C,
	numbers=left,
%	linewidth=10cm,
	breaklines}
\begin{lstlisting}
#include <stdio.h>
#include <stdlib.h>
#include <fcntl.h>
#include <sys/poll.h>
#include <unistd.h>

int main(void)
{
	struct pollfd pfd;
	char buffer;
	system("echo 3 > /sys/class/gpio/export");
	system("echo falling > /sys/class/gpio/gpio3/edge");
	system("echo in > /sys/class/gpio/gpio3/direction");
	pfd.fd = open("/sys/class/gpio/gpio3/value", O_RDONLY);
	if(pfd.fd < 0)
	{
		puts("Erro abrindo /sys/class/gpio/gpio3/value");
		puts("Execute este programa como root");
		return 1;
	}
	read(pfd.fd, &buffer, 1);
	pfd.events = POLLPRI | POLLERR;
	pfd.revents = 0;
	puts("Augardando campainha");
	poll(&pfd, 1, -1);
	if(pfd.revents) puts("Campainha pressionada: iniciar rotina de reconhecimento");
	close(pfd.fd);
	system("echo 3 > /sys/class/gpio/unexport");
	return 0;
}

\end{lstlisting}


\noindent\rule{\columnwidth}{2pt}


\textit{abre.sh}:



\lstset{language=bash,
	numbers=left,
	linewidth=8cm,
	breaklines}
\begin{lstlisting}
#!/bin/bash

GPIO_PATH=/sys/class/gpio

omxplayer -o local /home/pi/embarcados/projeto_final/sons/sim.mp3
echo 4 >> $GPIO_PATH/export
sudo echo out > $GPIO_PATH/gpio4/direction
sudo echo 1 > $GPIO_PATH/gpio4/value
sleep 3
echo 0 > $GPIO_PATH/gpio4/value
echo 4 >> $GPIO_PATH/unexport

\end{lstlisting}


\noindent\rule{\columnwidth}{2pt}


\textit{telegram.py}:
(C�digo do servidor para comunica��o com o bot do telegram)



\lstset{language=python,
	numbers=left,
	linewidth=8cm,
	breaklines}
\begin{lstlisting}
import sys
import time
import telepot
from telepot.loop import MessageLoop

def handle(msg):
    chat_id = msg['chat']['id']
    command =msg['text']
    print(command, chat_id)

#Inicio de conversa
    if command == 'Ola':
        bot.sendMessage(chat_id, 'Ol� Mestre, o que voc� deseja? ')
    if command == 'oi':
        bot.sendMessage(chat_id, 'Ol� Mestre, o que voc� deseja? ')
    if command == 'ola':
        bot.sendMessage(chat_id, 'Ol� Mestre, o que voc� deseja? ')
    if command == 'Oi':
        bot.sendMessage(chat_id, 'Ol� Mestre, o que voc� deseja? ')
# Cadastra novo usuario
    if command == 'Cadastrar':
        bot.sendMessage(chat_id, 'Informe o nome do usuario')

# Entrar
    if command == 'Abre':
        os.system('abre.sh')

    if command == 'Negado':
        os.system('negado.sh')

# Deletar usuario


bot = telepot.Bot('556806366:AAH9OIYZwwKapLkZrs7fYQU-NdUF2H9MuDc')
MessageLoop(bot, handle).run_as_thread()
print ('lendo ...')

# Keep the program running.
while 1:
    time.sleep(10)
    
\end{lstlisting}


\noindent\rule{\columnwidth}{2pt}


\textit{facerec$\_$from$\_$web.py}:
(C�digo do reconhecimento facial)



\lstset{language=python,
	numbers=left,
	linewidth=8cm,
	breaklines}
\begin{lstlisting}

import face_recognition
import cv2

# This is a demo of running face recognition on live video from your webcam. It's a little more complicated than the
# other example, but it includes some basic performance tweaks to make things run a lot faster:
#   1. Process each video frame at 1/4 resolution (though still display it at full resolution)
#   2. Only detect faces in every other frame of video.

# PLEASE NOTE: This example requires OpenCV (the `cv2` library) to be installed only to read from your webcam.
# OpenCV is *not* required to use the face_recognition library. It's only required if you want to run this
# specific demo. If you have trouble installing it, try any of the other demos that don't require it instead.

# Get a reference to webcam #0 (the default one)
video_capture = cv2.VideoCapture(0)

# Load a sample picture and learn how to recognize it.
obama_image = face_recognition.load_image_file("obama.jpg")
obama_face_encoding = face_recognition.face_encodings(obama_image)[0]

# Load a sample picture and learn how to recognize it.
luana_image = face_recognition.load_image_file("luana.jpg")
luana_face_encoding = face_recognition.face_encodings(luana_image)[0]

# Load a sample picture and learn how to recognize it.
vitinho_image = face_recognition.load_image_file("vitinho.jpg")
vitinho_face_encoding = face_recognition.face_encodings(vitinho_image)[0]

# Load a second sample picture and learn how to recognize it.

# Create arrays of known face encodings and their names
known_face_encodings = [
    obama_face_encoding,
    luana_face_encoding,
    vitinho_face_encoding
]
known_face_names = [
    "Barack Obama",
    "Luana",
    "vitinho"
]

# Initialize some variables
face_locations = []
face_encodings = []
face_names = []
process_this_frame = True

while True:
    # Grab a single frame of video
    ret, frame = video_capture.read()

    # Resize frame of video to 1/4 size for faster face recognition processing
    small_frame = cv2.resize(frame, (0, 0), fx=0.25, fy=0.25)

    # Convert the image from BGR color (which OpenCV uses) to RGB color (which face_recognition uses)
    rgb_small_frame = small_frame[:, :, ::-1]

    # Only process every other frame of video to save time
    if process_this_frame:
        # Find all the faces and face encodings in the current frame of video
        face_locations = face_recognition.face_locations(rgb_small_frame)
        face_encodings = face_recognition.face_encodings(rgb_small_frame, face_locations)

        face_names = []
        for face_encoding in face_encodings:
            # See if the face is a match for the known face(s)
            matches = face_recognition.compare_faces(known_face_encodings, face_encoding)
            name = "Unknown"

            # If a match was found in known_face_encodings, just use the first one.
            if True in matches:
                first_match_index = matches.index(True)
                name = known_face_names[first_match_index]

            face_names.append(name)

    process_this_frame = not process_this_frame


    # Display the results
    for (top, right, bottom, left), name in zip(face_locations, face_names):
        # Scale back up face locations since the frame we detected in was scaled to 1/4 size
        top *= 4
        right *= 4
        bottom *= 4
        left *= 4

        # Draw a box around the face
        cv2.rectangle(frame, (left, top), (right, bottom), (0, 0, 255), 2)

        # Draw a label with a name below the face
        cv2.rectangle(frame, (left, bottom - 35), (right, bottom), (0, 0, 255), cv2.FILLED)
        font = cv2.FONT_HERSHEY_DUPLEX
        cv2.putText(frame, name, (left + 6, bottom - 6), font, 1.0, (255, 255, 255), 1)

    # Display the resulting image
    cv2.imshow('Video', frame)

    # Hit 'q' on the keyboard to quit!
    if cv2.waitKey(1) & 0xFF == ord('q'):
        break

# Release handle to the webcam
video_capture.release()
cv2.destroyAllWindows()

\end{lstlisting}

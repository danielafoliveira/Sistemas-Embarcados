\section{Discuss�o e Conclus�es}

A arquitetura multi-thread permitiu que os perif�ricos fossem controlados simultaneamente. Isso � fundamental para o projeto, tanto do ponto de vista de experi�ncia do usu�rio, que n�o precisa esperar o t�rmino de alguma rotina para que a fun��o de interesse seja executada, quanto para a seguran�a do sistema, visto que h� uma vigilancia permanente do estado da porta.

Um dos problemas que a dupla teve durante o desenvolvimento do software, foi o compartilhamento de vari�veis entre as threads e processos pai e filho. Inicialmente a altera��o das vari�veis utilizadas como flags foi feita no processo pai, e as threads criadas no processo filho realizava as leituras. Por�m isso n�o funcionou, porqu� os processos n�o compartilham valores de vari�veis, apenas suas declara��es. Esse problema foi resolvido realizando todas as opera��es com vari�veis flags dentro do mesmo processo.

Para o encerramento do programa via comando CTRL+C, foi necess�rio utilizar a captura do sinal SIGINT e encaminhar para uma fun��o de encerramento. Esta realiza o cancelamento das threads e o 

O servidor � um programa separado, e n�o uma fun��o e nem um processo filho. Logo, para o c�digo principal receber comandos atrav�s do servidor, foi necess�rio utilizar m�todos de escrita em arquivo para comunicar os dois processos.


Uma limita��o das bibliotecas � que elas n�o diferenciam rostos reais de rostos em fotos mostradas para a c�mera. Isso � um grande problema de seguran�a para o projeto, por�m a dupla j� est� estudando t�cnicas de diferencia��o destes casos.

